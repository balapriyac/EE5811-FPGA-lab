\documentclass[journal,12pt,twocolumn]{IEEEtran}
\usepackage[utf8]{inputenc}
\usepackage{amssymb}
\usepackage{setspace}
\usepackage{gensymb}
\singlespacing
\usepackage[mathscr]{euscript}
\usepackage{textgreek}
\usepackage{textcomp}
\usepackage{amsmath}
\numberwithin{equation}{section}
\usepackage{mathrsfs}
\usepackage{txfonts}
\usepackage{stfloats}
\usepackage{bm}
\usepackage{cite}
\usepackage{hyperref}
\usepackage{cases}
\usepackage{subfig}
\usepackage{graphicx}
\usepackage{circuitikz}
\usepackage{tikz} 
\usetikzlibrary{karnaugh}
\usepackage{listings}
\usepackage{amsfonts}
\usepackage{longtable}
\usepackage{multirow}
\usepackage{tcolorbox}
\usepackage{geometry}
\geometry{
 a4paper,
 total={170mm,257mm},
 left=20mm,
 top=20mm,
 }
\usepackage{listings}
\lstset{
frame=single, 
breaklines=true,
columns=fullflexible
}

\def\putbox#1#2#3{\makebox[0in][l]{\makebox[#1][l]{}\raisebox{\baselineskip}[0in][0in]{\raisebox{#2}[0in][0in]{#3}}}}
     \def\rightbox#1{\makebox[0in][r]{#1}}
     \def\centbox#1{\makebox[0in]{#1}}
     \def\topbox#1{\raisebox{-\baselineskip}[0in][0in]{#1}}
     \def\midbox#1{\raisebox{-0.5\baselineskip}[0in][0in]{#1}}
\vspace{3cm}

\usepackage{karnaugh-map}
\usepackage{hyperref}

\renewcommand\thesection{\arabic{section}}
\renewcommand\thesubsection{\thesection.\arabic{subsection}}
\renewcommand\thesubsubsection{\thesubsection.\arabic{subsubsection}}

\renewcommand\thesectiondis{\arabic{section}}
\renewcommand\thesubsectiondis{\thesectiondis.\arabic{subsection}}
\renewcommand\thesubsubsectiondis{\thesubsectiondis.\arabic{subsubsection}}

\title{EE5811 - FPGA Lab Assignment 1}
\author{EE19MTECH11032 - Bala Priya C}
\date{December 2021}

\begin{document}

\maketitle
Download the LaTex code from:
\begin{lstlisting}
https://github.com/balapriyac/EE5811-FPGA-lab/tree/main/Assignment-1
\end{lstlisting}

\section{Question}

[CBSE 2013 Q6 (b)] Obtain the Boolean expression for the logic circuit shown below:

\begin{figure}[h!]
    \centering
    \begin{circuitikz}\draw
       
        (3,0.75) node[or port] (or1) {}
        (0,1) node[ieeestd not port, scale = 0.5] (not1) {}
        (0,-2) node[ieeestd not port, scale = 0.5] (not2) {}
        (3,-1.75) node[and port] (and1) {}
        (5,-0.75) node[or port] (or2) {}
        
        (or1.in 2) -- (and1.in 1)
        (or1.out) -- (or2.in 1)
        (and1.out) -- (or2.in 2)
        (not1.out) -- (or1.in 1)
        (not2.out) -- (and1.in 2);
       
        
        \node(X) at (-2.5,1) {$X$};
        \node(Y) at (-2.5,0.5) {$Y$};
        \node(Z) at (-2.5,-2) {$Z$};
        \node(F) at (5.5,-0.75){$F$};
        
        \draw (-2.2,1) |- (not1.in);
        \draw (-2.2,0.5) |- (or1.in 2);
        \draw (-2.2,-2) |- (not2.in);
        
\end{circuitikz}

    \caption{Logic circuit}
    \label{fig:circuit}
\end{figure}

\section{Solution}
The algebraically simplified Boolean expression for F is shown below.
\begin{align}
F = (X' + Y) + YZ' \\
= X' + (Y + YZ')\\
= X' + Y(1 + Z')\\
= X' + Y(1 + Z')\\
= X' + Y  \;\; since \; 1 + Z' = 1\\
\implies F = X' + Y
\end{align}

\section{Truth Table}
The truth table corresponding to F for various choices of input X, Y, Z is shown below:
\medskip

    \centering
          \begin{tabular}{|c|c|c|c|}
            \hline
            X & Y & Z & F \\
            \hline
            0 & 0 & 0 & 1\\
            \hline
             0 & 0 & 1 & 1\\
            \hline
             0 & 1 & 0 & 1\\
            \hline
             0 & 1 & 1 & 1\\
            \hline
            1 & 0 & 0 & 0\\
            \hline
            1 & 0 & 1 & 0\\
            \hline
            1 & 1 & 0 & 1\\
            \hline
            1 & 1 & 1 & 1\\
            \hline
            
        \end{tabular}
     

\section{Simplification Using Karnaugh Map}


\begin{figure}[h!]
    \centering
    \begin{karnaugh-map}[4][2][1][][]
        
        \minterms{0,1,2,3,6,7}
        \maxterms{4,5}
        \implicant{0}{2}
        \implicant{3}{6}
        \draw[color=black, ultra thin] (0, 2) --
        node [pos=0.7, above right, anchor=south west] {$YZ$} % Y label
        node [pos=0.7, below left, anchor=north east] {$X$} % X label
        ++(135:1);
        
    \end{karnaugh-map}
    \caption{K-Map for $F = X' + Y$}
    \label{fig:kmap}
\end{figure}

\section{Implementation Using NAND Logic}
\begin{align}
F = X' + Y \\
= (X.Y')' \\
since \; \;  (A.B)' = A' + B')\\
\implies F = (X.Y')' \;(using \;NAND\; logic)
\end{align}

\begin{figure}[h!]
    \centering
    \begin{circuitikz}\draw
       
        (3,0) node[nand port] (nand2) {}
        (1,-0.3) node[nand port, scale = 0.5] (nand1) {}
        
        (nand1.out) -- (nand2.in 2)
        (nand1.in 1) -- (nand1.in 2);
        
        \node(X) at (-1,0.25) {$X$};
        \node(Y) at (-1,-0.5) {$Y$};
        \node(F) at (3.5,0){$F$};
        
        \draw (-0.8,0.25) |- (nand2.in 1);
        \draw (-0.8,-0.3) |- (0.3,-0.3);
        
    \end{circuitikz}
   
    \caption{Logic circuit using NAND logic}
    \label{fig:NAND logic ckt}
\end{figure}

\end{document}
